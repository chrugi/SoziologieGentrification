% Created 2018-05-02 Mit 17:46
\documentclass[a4paper,ngerman,11pt]{scrartcl}
\usepackage[utf8]{inputenc}
\usepackage[T1]{fontenc}
\usepackage{fixltx2e}
\usepackage{graphicx}
\usepackage{longtable}
\usepackage{float}
\usepackage{wrapfig}
\usepackage{rotating}
\usepackage[normalem]{ulem}
\usepackage{amsmath}
\usepackage{textcomp}
\usepackage{marvosym}
\usepackage{wasysym}
\usepackage{amssymb}
\tolerance=1000
\usepackage{natbib}
\usepackage[linktocpage,pdfstartview=FitH,colorlinks,linkcolor=black,
anchorcolor=black,citecolor=black,filecolor=black,menucolor=black,urlcolor=black]{hyperref}
\addtokomafont{disposition}{\rmfamily}
\usepackage{ngerman}
\usepackage{url}
\author{Christian Sangvik}
\date{2. Mai 2018}
\title{Qualitäten allgemein Buchegg}
\hypersetup{
  pdfkeywords={},
  pdfsubject={},
  pdfcreator={Emacs 25.3.1 (Org mode 8.2.10)}}
\begin{document}

\maketitle
\noindent{}
Was die Qualitäten des Quartiers um den Bucheggplatz sind, mag auf den ersten
Blick nicht gleich ersichtlich sein. Das ganze Gebiet scheint von der
Blechlawine des Privatverkehrs regelrecht überrollt und verschluckt zu werden.

Trotzdem wollen die Leute hier nicht weg. Keiner unserer Interviewpartner hegte
den Wunsch, woanders zu wohnen, denn die Qualitäten überwiegen, versicherte man
uns in jedem Gespräch. Sobald man sich ein paar Meter von den Hauptachsen
wegbewegt können wir das auch nachvollziehen. Es zeigt sich ein komplett anderes
Bild als zuvor. Das Quartier hat offensichtlich auch seine ruhigen und
gemütlichen Orte und entwickelt so einen eigentümlich einladenden und fast
intimen Charakter. In den diversen Innenhöfen der angrenzenden Überbauungen
findet ein anderes Leben statt, als wir erwartet haben.

Die Bewohner schätzen es sehr, dass sie auf der einen Seite fast wohnen, wie auf
dem Land; mit ruhigen Quartiersstrassen, wo die Kinder draussen spielen können,
ohne dass man gleich Angst haben muss, dass ihnen etwas zustösst. Der sehr nahe
gelegenen Wald und die Schrebergärten im Waid dienen als Naherholungsgebiet. Auf
der anderen Seite ist man hier phänomenal an den Rest der Stadt angebunden, egal
ob mit Bus, Tram oder dem Auto. Diese Zentralität schätzen natürlich auch die
Gewerbetreibenden.

Die vielen Genossenschaften des Quartiers begünstigen auch das Sozialleben. Die
Bewohner scheinen ein gutes Verhältnis und wesentlich mehr Kontakt untereinander
zu haben, als es für eine Stadt zu erwarten ist. So sind denn die Höfe und
Freiflächen sehr wichtig für das Alltagsleben im Quartier.
% Emacs 25.3.1 (Org mode 8.2.10)
\end{document}
