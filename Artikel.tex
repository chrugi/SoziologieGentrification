% Created 2018-04-27 Fre 13:23
\documentclass[a4paper,ngerman,11pt]{scrartcl}
\usepackage[utf8]{inputenc}
\usepackage[T1]{fontenc}
\usepackage{fixltx2e}
\usepackage{graphicx}
\usepackage{longtable}
\usepackage{float}
\usepackage{wrapfig}
\usepackage{rotating}
\usepackage[normalem]{ulem}
\usepackage{amsmath}
\usepackage{textcomp}
\usepackage{marvosym}
\usepackage{wasysym}
\usepackage{amssymb}
\tolerance=1000
\usepackage{natbib}
\usepackage[linktocpage,pdfstartview=FitH,colorlinks,linkcolor=black,
anchorcolor=black,citecolor=black,filecolor=black,menucolor=black,urlcolor=black]{hyperref}
\usepackage{ngerman}
\addtokomafont{disposition}{\rmfamily}
\author{Gruppe 1}
\date{27. April 2018}
\title{Artikel}
\hypersetup{
  pdfkeywords={},
  pdfsubject={},
  pdfcreator={Emacs 25.3.1 (Org mode 8.2.10)}}
\begin{document}

\maketitle
\tableofcontents


\section{{\bfseries\sffamily TODO} Gemeinsamer Text (6000 - 7000 Zeichen ohne Leerzeichen)}
\label{sec-1}

Text verdichtet Material aus Beobachtungsprotokoll..Interviews

\section{{\bfseries\sffamily TODO} 20 Fotos}
\label{sec-2}


\section{These}
\label{sec-3}

Es gibt im Quartier zwar Treffpunkte, aber keine Aufenthaltsorte.

\section{Text}
\label{sec-4}

\subsection{Treffpunkte ohne Aufenthaltsorte}
\label{sec-4-1}

\subsubsection{Qualität der Aufenthaltsorte}
\label{sec-4-1-1}

\subsubsection{Falsche Zielgruppe?}
\label{sec-4-1-2}

\section{Fragestellung}
\label{sec-5}

\begin{itemize}
\item Wie gestalten sich das alltägliche städtische Leben entlang der
Rosengartenstrasse und im Quartier?

\item Was zeichnet das Quartier aus?

\item Welche urbanen Qualitäten schätzen die Bewohner und Gewerbebetreibenden?

\item Was würde eine Verkehrsberuhigung für sie bedeuten?
\end{itemize}

\section{Struktur}
\label{sec-6}

\subsection{Verkehrsknotenpunkt}
\label{sec-6-1}

\subsection{Container}
\label{sec-6-2}

\subsubsection{Geschichte des Containers}
\label{sec-6-2-1}

sehr kurz

\subsubsection{Was war vorher, wie verändert der Container die Qualität?}
\label{sec-6-2-2}

Leute waren nicht da, wie kommen sie jetzt an den Ort?

\subsubsection{Qualitäten allgemein}
\label{sec-6-2-3}

\subsection{Was macht einen Aufenthaltsort aus?}
\label{sec-6-3}

\subsubsection{Wie hebt sich der Container als Aufenthaltsort aus im vergleich zu anderen Treffpunkten im Quartier?}
\label{sec-6-3-1}

Aufenthaltsort vs. Treffpunkt, vage diskutiert, konkret am Beispiel
Container aufgezeigt.

\subsection{These (anhand von Ort, konkret)}
\label{sec-6-4}

\subsection{Treffpunkt Container}
\label{sec-6-5}

\subsection{Alternativen und Ergänzungen im Quartier}
\label{sec-6-6}

\subsubsection{GZ Buchegg erwähnen}
\label{sec-6-6-1}

\subsection{Innenhöfe und halbprivate Treffpunkte}
\label{sec-6-7}
% Emacs 25.3.1 (Org mode 8.2.10)
\end{document}
