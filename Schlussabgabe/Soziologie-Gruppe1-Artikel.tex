\documentclass[a4paper,ngerman,11pt]{scrartcl}

\usepackage{ngerman}
\usepackage{booktabs}

\usepackage[utf8]{inputenc}
\usepackage[T1]{fontenc}

\renewcommand{\arraystretch}{2}

\begin{document}

\begin{center}
  \begin{tabular}{{c}}
    \toprule
    \large{Soziologie | Gruppe 1}\\
    \Huge{\uppercase{Treffpunkt am}}\\
    \Huge{\uppercase{Bucheggplatz}}\\
    \bottomrule
\end{tabular}
\end{center}

\vspace*{2cm}

\noindent{}
„Nächster Halt, Bucheggplatz“, sagt die bekannte Stimme aus dem Lautsprecher des
72er-Trolleybusses Richtung Milchbuck. Wie fast jeden Tag während des Semesters
fahre ich diese Strecke. Ständig ändert sich etwas entlang der Strasse, viele
Häuser werden umgebaut oder abgerissen. Das Einzige was konstant bleibt, ist der
Verkehr und die Baustellen.

Ich steige aus und schaue mir als erstes den Innenhof der neuen Überbauung am
Bucheggplatz an. Die Grünarbeiten sind noch nicht ganz fertig, die Kinder
spielen aber bereits auf dem Spielplatz und die neuen Bewohner geniessen die
ersten Sonnenstrahlen auf ihren Balkonen. Vom Verkehrslärm ist wenig zu
hören. Es kommt einem vor wie in einer eigenen Welt, nur die Aussicht auf den
Prime Tower und das Swissmill-Silo erinnern an die Grossstadt.

Über einen Ausläufer der Fussgängerbrücke, der Spinne, wie sie auch genannt
wird, gelange ich ins Zentrum des Bucheggplatzes. Von dort oben sieht man die
vielen Gebäude, die neu gebaut wurden und noch eingekleidet sind.

Unten angekommen ist der laute Verkehr und das ständige Abbremsen und Anfahren
der Busse und Trams allgegenwärtig.

Einen ruhigen Platz für meine Beobachtung zu suchen ist nicht so einfach. Vor
ein paar Jahren wurde der Bucheggplatz durch verschiedene rote Bänke und
Blumenbeete ergänzt. Auf einem dieser Bänke mache ich es mir gemütlich und
betrachte das hektische Treiben.

Der grösste Teil der Leute verbringt nur etwa fünf Minuten am Bucheggplatz. Die
Pendler starren auf ihre Mobiltelefone und warten ungeduldig auf ihre nächste
Verbindung. Oder sie versuchen rennend, ihre Anschlussverbindung noch zu
erwischen. Einige wenige steigen hier aus, gehen sogleich über die Strasse und
verschwinden in ihren Häusern.

Was jedoch gar nicht ins Bild dieses Pendlerumschlagplatzes passt, ist das
gemütliche Café, das seit ein paar Monaten in einem alten Schiffscontainer
betrieben wird. Der rote Container sieht aus, als wäre er vor vielen Jahren
direkt mit der roten Spinne zusammen geplant worden. An den Tischen davor sitzen
junge Frauen mit Kinderwagen, die in Gespräche vertieft sind und den ganzen
Trubel um sie herum gar nicht wahrnehmen. Es sieht aus, als wären sie Anwohner.

Solche Orte zum Treffen und Verweilen gibt es nicht viele in der Nähe. Einige
kaufen aber auch nur kurz einen Kaffee zum Mitnehmen und gehen dann sofort
weiter.

Was auffällt: Nur ältere Leute setzen sich auf die roten Bänke und lesen
gemütlich eine Zeitung oder beobachten alles neugierig. Nach einer halben Stunde
wird es aber auch ihnen zu laut und sie gehen weiter.

Einige Taxifahrer warten auf Kunden, doch an diesem Verkehrsknoten-punkt
brauchen wenige ihre Dienste.

Die Interaktionen zwischen den meisten Personen beschränken sich auf ein
Minimum, was will man auch gross miteinander reden bei dem Lärm.

Langsam wird es auch mir zu laut hier und ich steige in den nächsten 69er-Bus
Richtung ETH Hönggerberg.“

\section*{Alltagsleben im Quartier}

Am Bucheggplatz, wo die Strassen, Tram- und Buslinien zusammentreffen scheint
sich das gesamte Strassenleben des Quartieres zu konzentrieren. Bei genauerer
Betrachtung stelle ich aber fest, dass die meisten Leute lediglich von einer
ÖV-Linie auf die andere umsteigen. Es scheint als ob der gesamte öffentliche
Raum im Quartier einzig darauf ausgelegt sei.

Von hier weg führt die Bucheggstrasse, welche mit ihren viel befahrenen vier
Spuren das Quartier entzwei teilt. Eine enge Unterführung ist der einzige, wenn
auch klägliche, Versuch, die zwei Teile irgendwie für den Langsamverkehr zu
verbinden; Fussgängerstreifen gibt es nicht.

Geht man von der Hauptstrasse weg ins Quartier hinein, wandelt sich die Stimmung
schlagartig. Der Verkehrslärm ist verschwunden, ebenso die urbane
Atmosphäre. Zwischen den Häusern liegen grosse, grüne Gärten und die neueren
Siedlungen warten mit belebten Innenhöfen auf, wo Kleinkinder sich auf
Spielplätzen vergnügen und ihre jungen Eltern miteinander plaudern.

Im starken Kontrast dazu steht das Strassenleben. Niemand scheint wirklich im
öffentlichen Raum zu verweilen. Am belebtesten wirkt noch die Bushaltestelle an
der Lägernstrasse, wo man grosse Gruppen junger Männer beobachten kann, die
sich hier treffen, um gemeinsam auf den Bus zu warten und weg zu fahren.

\section*{Der Container}

Was die wahren Qualitäten des Quartiers um den Bucheggplatz sind, mag auf
den ersten Blick immer noch nicht gleich ersichtlich sein. Das ganze Gebiet
scheint von der Blechlawine des Privatverkehrs regelrecht überrollt und
verschluckt zu werden.

Trotzdem wollen die Leute hier nicht weg. Keiner unserer Interviewpartner hegte
den Wunsch, woanders zu wohnen, denn die Qualitäten überwiegen, versicherte man
mir in jedem Gespräch.

Sobald man sich ein paar Meter von den Hauptachsen wegbewegt, kann man das auch
nachvollziehen. Es zeigt sich ein komplett anderes Bild als zuvor. Das Quartier
hat offensichtlich auch seine ruhigen und gemütlichen Seiten und entwickelt so
einen eigentümlich einladenden und fast intimen Charakter.


In den diversen Innenhöfen der angrenzenden Überbauungen findet ein anderes
Leben statt, als ich erwartet habe. Man wähnt sich am Stadtrand oder – vor allem
im nördlichen Quartiersteil – in einem Dorf.  Die Bewohner schätzen es sehr,
dass sie auf der einen Seite fast wohnen, wie auf dem Land. Mit ruhigen
Quartiersstrassen, wo die Kinder draussen spielen können, ohne dass man gleich
Angst haben muss, dass ihnen etwas zustösst.

Der sehr nahe gelegenen Wald und die Schrebergärten im Waid dienen als
Naherholungsgebiet. Auf der anderen Seite ist man hier optimal an den Rest der
Stadt angebunden, egal ob mit Bus, Tram oder dem Auto. Diese Zentralität
schätzen natürlich auch die Gewerbetreibenden.

Die vielen Genossenschaften des Quartiers begünstigen auch das Sozialleben. Die
Bewohner scheinen ein gutes Verhältnis und wesentlich mehr Kontakt untereinander
zu haben, als es für eine Stadt zu erwarten ist. So sind denn die Höfe und
Freiflächen sehr wichtig für das Alltagsleben im Quartier. Und mit dem Container
gibt es jüngst einen neuen Vorstoss in Richtung öffentlichem Sozialleben.

Inspiriert von der „kleinen Freiheit“ an der Haldenegg, haben sich im Jahr 2013
drei engagierte Zürcherinnen aus der unmittelbaren Nachbarschaft des
Bucheggplatzes dafür eingesetzt, ein ähnliches Projekt an eben diesem Platz
umzusetzen.

Nach anfänglichen Rückschlägen und einem langen Prozedere gelang es im Dezember
2017 schliesslich, das Vorhaben der kleinen kulinarischen Oase am Bucheggplatz
zu realisieren. Da das Café in einem umfunktionierten Schiffscontainer Platz
finden sollte, stand das „Kumo6“ innerhalb kürzester Zeit auf dem Platz.

Einige der roten Bänke und Sitzgelegenheiten am Bucheggplatz wurden dem Platz
erst kürzlich im Jahr 2014 hinzugefügt. Zudem erfuhr der Platz an einigen
Stellen eine Neugestaltung.

Das Bedürfnis, dem dynamischen Umschlagplatz einen Gegenpol in Form von
Aufenthaltsmöglichkeiten zu setzen, wurde dadurch allerdings noch nicht gänzlich
befriedigt. Vor allem den Anwohnern fehlte es weiterhin an Orten der
Interaktion. Befragte Pendler wie auch Anwohner sehnen sich nach mehr Cafés,
Bars, Restaurants und Einkaufsmöglichkeiten am Bucheggplatz. Das Angebot mit
einem Denner, Kiosk und einem Blumenladen ist sehr überschaubar.

Der erst kürzlich neugestaltete Platz an der Ecke Hofwiesenstrasse wird durch
das „Kumo6“ nun tatsächlich auch belebt und stellt für Pendler und Anwohner eine
erfreuliche Abwechslung dar.

Durch die Setzung des Containers stehen die Bänke heute auch an einem etwas
ruhigeren Ort, abgeschirmt von der Strasse. Kaffee und Sandwich laden ein, seine
Mittagspause an einem der Tische zu verbringen. Oder man geniesst ganz einfach
sein Feierabendbier auf einer der Liegen.

Ausserhalb der Pendler-Rushhour wird der Container zu einem beliebten Treffpunkt
für direkte Anwohner, das können sowohl junge Eltern mitsamt Kindern, als auch
Rentner sein. Die Geschäftsführer beschreiben ihre Kundschaft als sehr
durchmischt. Oft sind es auch Studenten oder Leute die zur Arbeit gehen und sich
einfach rasch einen Kaffee auf den Weg holen.

Ob man sich jetzt für eine halbe Stunde hinsetzt oder doch nur was für mit auf
den Weg holt - das „Kumo6“ haucht dem Bucheggplatz neues Leben ein und könnte
Inspiration und Vorreiter für weitere neue Ideen der Platzbespielung sein.

\section*{Fazit}

Abschliessend kann man sagen, dass auch ich überrascht wurde von der
vorgefundenen Vielfalt im Quartier. So eintönig wie es auf der einen Seite zu
sein scheint, so facettenreich zeigt es sich auf der anderen. Der Verkehrslärm
ist schnell vergessen, wagt man sich ein bisschen tiefer ins Quartier
hinein. Mit dem „Kumo 6“ wurde der erste Stein gelegt, um den Bucheggplatz
sowohl für die Anwohner, als auch für die Pendler attraktiver zu gestalten und
macht Hoffnung auf mehr.

\end{document}
